\chapter{L\'evy Processes}
\label{sec:Levy}

\cleanchapterquote{Paul L\'evy was a painter in the probabilistic world.}{Michel Lo\`eve}{(1907-1979)}

The L\'evy processes play a central role in mathematical finance. They can describe the reality of financial markets in more accurate way than models based on the geometric Brownian motion used in particular in Black-Scholes model. Indeed we can observe in the real world that the asset price processes have jumps. Moreover, the log returns of the underlying have empirical distribution with fat tails and skewness which deviates from normality supposed by Black and Scholes. We will begin this chapter, in section \ref{sec:Levy:definitions}, with the definition of a L\'evy process and expose its fundamental properties.

\section{Definitions and properties}
\label{sec:Levy:definitions}
The L\'evy processes, which are the continuous-time case of random walks, are ingredients for building continuous-time stochastic models. The Poisson process and the Wiener process are the most famous examples of L\'evy processes. We will see later in this chapter that every L\'evy process is a superposition of a Wiener process and a (possibly infinite) number of independent Poisson processes. Let us define these two processes.

\begin{defn}[Wiener Process]\label{def:wiener}
A stochastic process $W = \{W_t,t\geq 0\}$, with $W_0=0$, is a \textbf{Wiener process}, also called a standard Brownian motion, on a probability space $(\Omega,\mathcal{F},\mathbb{P})$ if:
\begin{my_list_num}
\item $W$ has independent increments, i.e. $(W_{t+s}-W_t)$ is independent of $\mathcal{F}_t$ for any $s>0$.
\item $W$ has stationary increments, i.e. the distribution of $(W_{t+s}-W_t)$ does not depend on $t$.
\item $W$ has Gaussian increments, i.e. $(W_{t+s}-W_t) \sim \mathcal{N}(0,s)$.
\item $W$ is stochastically continuous, i.e. $$\forall \epsilon>0: \lim_{s \to t}\mathbb{P}(|W_t-W_s|<\epsilon)=0.$$
\end{my_list_num}
\end{defn}

This motion was discovered by Brown in 1827 and taken back by \citeauthor{Bachelier} \citeyearpar{Bachelier} to model the stock market prices. Only in 1923 the Brownian was defined and constructed rigorously by R. Wiener.

\begin{defn}[Poisson process]\label{def:poisson}
Let $(\tau_i)_{i\geq 1}$ be a sequence of independent exponential random variables with parameter $\lambda$ and $T_n = \sum_{i=1}^n \tau_i$. The process $N = \{N_t,t\geq 0\}$, with $N_0=0$, defined by
$$N_t = \sum_{n\geq 1}\mathbf{1}_{\{t\geq T_n\}}$$
is called \textbf{Poisson process} with intensity $\lambda$.

This process has the following properties:
\begin{my_list_num}
\item $N$ has independent increments, i.e. $(N_{t+s}-N_t)$ is independent of $\mathcal{F}_t$ for any $s>0$. 
\item $N$ has stationary increments, i.e. the distribution of $(N_{t+s}-N_t)$ does not depend on $t$.
\item $N$ has Poisson increments, i.e. $(N_{t+s}-N_t)$ has a Poisson distribution with parameter $\lambda s$.
\item $N$ is stochastically continuous, i.e. $$\forall \epsilon>0: \lim_{s \to 0}\mathbb{P}(|N_{t+s}-N_t|<\epsilon)=0.$$
\end{my_list_num}
When the process is characterized by a constant intensity parameter $\lambda$, we say that the process is homogeneous. If the intensity parameter varies with time $t$ as $\lambda(t)$, the process is said to be non-homogeneous.
\end{defn}

The Poisson process, which bears the name of the French physicist and mathematician Sim\'eon Denis Poisson, defines a counting process. It counts the number of random times $(T_n)$ which occur in $[0,t]$. Therefore, this is an increasing pure jump process. The jumps of size 1 occur at times $T_n$ and the intervals between two jumps are exponentially distributed. If we compare definitions \ref{def:wiener} and \ref{def:poisson}, we can see that only property 4 differs between the two processes, only the distribution changes. The main idea of a L\'evy process is to ignore the distribution of increments.

\begin{defn}[L\'evy process]
A cadlag stochastic process $X =\{X_t,t\geq 0\}$ on $(\Omega,\mathcal{F},\mathbb{P})$ with real values is called a \textbf{L\'evy process} if it has the following properties:
\begin{my_list_num}
\item $X$ has independent increments, i.e. $(X_{t+s}-X_t)$ is independent of $\mathcal{F}_t$ for any $s>0$. 
\item $X$ has stationary increments, i.e. the distribution of $(X_{t+s}-X_t)$ does not depend on $t$. 
\item $X$ is stochastically continuous, i.e. $\forall \epsilon>0: \lim_{s \to 0}\mathbb{P}(|X_{t+s}-X_t|<\epsilon)=0.$
\end{my_list_num}
\end{defn}

The third condition does not imply that the sample paths are continuous. In fact the Brownian motion is the only (non-deterministic) L\'evy process with continuous sample paths. This condition serves to exclude jumps at non-random times. In other words, for a given $t$, the probability of seeing a jump at $t$ is zero, discontinuities occur at random time.

Let us now discuss the relationship between infinitely divisible distributions and L\'evy process.

\begin{defn}[Infinite divisibility]
A probability distribution $F$ is said to be \textbf{infinitely divisible} if for any integer $n\geq 2$, there exists $n$ i.i.d. random variables $Y_1,\ldots,Y_n$ such that $Y_1+\cdots+Y_n$ has distribution $F$.
\end{defn}
If $X$ is a L\'evy process, for any $t>0$ the distribution of $X_t$ is infinitely divisible. This comes from the fact that for any $n \geq 1$,
\begin{equation}\label{eq:inf-div}
X_t = X_{t/n} + (X_{2t/n}-X_{t/n}) + \cdots + (X_t - X_{(n-1)t/n}),
\end{equation}
and the property of stationary and independent increments. Let define now the characteristic function and characteristic  exponent of $X_t$.

\begin{defn}[Characteristic function and exponent]
The \textbf{characteristic function} $\Phi_t$ of a random variable $X_t$ with cumulative distribution $F_t$ is given by
$$\Phi_t(u) = \mathbb{E}\left[e^{iu X_t}\right]=\int_{-\infty}^{\infty}e^{i\theta x}dF_t(x).$$
Its \textbf{characteristic exponent} is given by
$$\Psi_t(u) = -\log\left(\mathbb{E}\left[e^{iu X_t}\right]\right),$$
for $u \in\mathbb{R}$ and $t>0$.
\end{defn}

 Then using twice equation \eqref{eq:inf-div} we obtain for any positive integers $m,n$ that
$$m\Psi_1(u) = \Psi_m(u) = n \Psi_{m/n}(u).$$
Hence for any rational $t=\frac{m}{n}>0$ we have
$$\Psi_t(u) = t\Psi_1(u).$$
We can generalize this relation for all $t>0$ with the help of the almost sure continuity of $X$ and a sequence of rational $\{t_n, n\geq 1\}$ such that $t_n\downarrow t$.

In conclusion, any L\'evy process has the property that for all $t>0$
$$\mathbb{E}\left[e^{iu X_t}\right] = e^{-t\Psi(u)},$$
where $\Psi(u) = \Psi_1(u)$ is the characteristic exponent of $X_1$.

Then it is clear that each L\'evy process has an infinitely divisible distribution. This allows us to apply the celebrated L\'evy-Khinchine formula. 

\begin{thm}[Lévy-Khintchine formula]
Each L\'evy process can be characterized by a triplet $(\mu,\sigma,\nu)$ with $\mu \in \mathbb{R},\sigma \geq 0$ and $\nu$ a measure satisfying $\nu(0) = 0$ and
$$\int_\mathbb{R} \min\{1,|x|^2\}\nu(dx)<\infty.$$
In term of this triplet the characteristic function of the L\'evy process equals:
\begin{align}\label{eq:LK}
\Phi(u) &= \mathbb{E}\left[\exp(i u X_t)\right]\nonumber\\
&= \exp\left(t\left(i\mu u -\frac{1}{2}\sigma^2u^2"\int_\mathbb{R}\left(e^{iux}-1-iux\mathbf{1}_{\{|x|<1\}}\nu(dx)\right)\right)\right).
\end{align}
(The proof can be find in \citeauthor{Tankov} \citeyearpar{Tankov})
\end{thm}

At the end of the story, we can conclude from equation \eqref{eq:LK} that every L\'evy processes are a combination of three independent processes: a linear deterministic process where $\mu$ is called the \textit{drift} term, a Wiener process with a \textit{diffusion} coefficient $\sigma$ and a \textit{pure jump} process whose dynamics is dictated by the L\'evy measure $\nu(dx)$. In fact the measure $\nu(dx)$ defines how happen the jumps, which occur according to a compound Poisson process with intensity $\lambda = \int_\mathbb{R}\nu(dx)$.

