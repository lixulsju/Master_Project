
% !TEX root = ../thesis-example.tex
%
\pdfbookmark[0]{Abstract}{Abstract}
\chapter*{Abstract}
\label{sec:abstract}

The FX Target Accrual Redemption Notes (FX-TARN) are financial products on currency pairs. These are speculation products of a very risky nature, which makes them very popular in Asia. 

This thesis presents different numerical methods for pricing FX-TARN under L\'evy processes. The main goal is to go beyond the famous Black-Scholes model, which does not perform good with respect to the market structure. Thus, L\'evy processes including jumps are more adapted to the market observations.

In general, options with path dependents payoff, such as this product, are evaluated by Monte Carlo simulations. We will describe two other methods based on Finite Difference (FD) and Fast Fourier Transform (FFT) in order to boost the performances of our pricing engine. At the end of this project, we will be able to propose a fast and accuracy method, which is not available in the literature, to price FX-TARN. This method is a combination of the methods proposed by \citeauthor{LS15} \citeyearpar{LS15} \cite{LS15} and \citeauthor{Lor+08} \citeyearpar{Lor+08} \cite{Lor+08}. 

The main advantage of the method, proposed in this work, is that it is easy to implement and allows us to extend it to all L\'evy processes with closed form characteristic function.

\chapter*{Résumé}

Les FX Target Accrual Redemption Notes (FX-TARN) sont des produits financier sur des paires de devises. Ce sont des produits de spéculation à caractère très risqué, ce qui les rend très populaire en Asie.

Cette thèse présente différentes méthodes numériques pour l'évaluation du prix d'un FX-TARN à l'aide de processus de Lévy. L'objectif principal est d'aller au-delà du célèbre modèle de Black-Scholes, qui ne performe pas bien par rapport à la structure du marché. Ainsi, les processus Lévy avec sauts sont plus adaptés aux observations du marché.

En général, les options avec un profit fortement dépendant de l'évolution du cours de la monnaie, comme ce produit, sont évaluées avec des simulations de Monte Carlo. Nous allons décrire deux autres méthodes basée sur les Différences Finies (FD) et les Transformées de Fourier (FFT) dans le but d'améliorer les perfomances de notre outil d'évaluation. \'A la fin de ce projet, nous serons en mesure de proposer une méthode rapide et précise, qui n'existe pas dans la litérature, pour évaluer les FX-TARN. Cette méthode est une combinaison des méthodes proposées par \citeauthor{LS15} \citeyearpar{LS15} \cite{LS15} et \citeauthor{Lor+08} \citeyearpar{Lor+08} \cite{Lor+08}. 

Le principal avantage de la méthode proposée dans ce travail, est qu'elle est facile à implémenter et nous permets de l'étendre à tous les modèles de Lévy avec une fonction caractéristique sous forme close.

