\chapter{Finite Difference Method}
\label{sec:FD}

\cleanchapterquote{Citation}{Author}{(Author description)}

Section Introduction

\section{Taylor Expansion}
\label{sec:FD:Taylor}
%Taylor Expansion
Recall that the Taylor expansion for a function $f\in C^\infty$ infinitely many differentiable is given by
\begin{align*}
f(x) &= f(a) + (x-a)\frac{\partial f}{\partial x}(x) + \frac{(x-a)^2}{2!}\frac{\partial^2 f}{\partial x^2}(x) + \cdots + \frac{(x-a)^n}{n!}\frac{\partial^n f}{\partial x^n}(x)+\cdots\\
&=\sum_{k = 0}^\infty \frac{(x-a)^k}{k!}\frac{\partial^k f}{\partial x^k}(x).
\end{align*}

If $f$ is only $(n+1)$ times continuously differentiable, i.e. $f\in C^{(n+1)}$, we can write
$$f(x)=\sum_{k = 0}^{n+1} \frac{(x-a)^k}{k!}\frac{\partial^k f}{\partial x^k}(x)+O((x-a)^{n+1}),$$
where $O((x-a)^{n+1})$ represents the remainder in Landau notation.

\subsection{Forward and Backward Difference Approximation of First Derivative}
%Forward and Backward Difference Approximation of First Derivative
In order to approximate $\frac{\partial f}{\partial t}(x)$ assume that $f$ is twice continuously differentiable, i.e. $f\in C^2$.  By a first order Taylor expansion we can write
\begin{equation}\label{forwardTaylor}
f(x + h) =  f(x) + h \frac{\partial f}{\partial x}(x)+ O(h),
\end{equation}
\begin{equation}\label{backwardTaylor}
f(x - h) =  f(x) -h \frac{\partial f}{\partial x}(x)+ O(h).
\end{equation}

The equation \eqref{forwardTaylor} gives us
\begin{equation}\label{fwd}
\frac{\partial f}{\partial x}(x) =  \frac{f(x+h)-f(x)}{h} + O(h),
\end{equation}
which is known as \textit{forward difference} approximation of the first derivative. 

On the other hand, the equation \eqref{backwardTaylor} gives us
\begin{equation}\label{bwd}
\frac{\partial f}{\partial x}(x) =  \frac{f(x)-f(x-h)}{h} + O(h),
\end{equation}
which is known as \textit{backward difference} approximation of the first derivative.

\subsection{Central Difference Approximation of First Derivative}
%Central Difference Approximation of First Derivative
Now assume that $f\in C^3$. Then with a second order Taylor expansion, we have
\begin{equation}\label{forwardTaylor2}
f(x) = f(x + h) + h \frac{\partial f}{\partial x}(x) + \frac{h^2}{2}\frac{\partial^2 f}{\partial x^2}(x) +O(h^2),
\end{equation}
\begin{equation}\label{backwardTaylor2}
f(x) = f(x - h) - h \frac{\partial f}{\partial x}(x) + \frac{h^2}{2}\frac{\partial^2 f}{\partial x^2}(x) +O(h^2).
\end{equation}

Subtracting equation \eqref{backwardTaylor2} from  \eqref{forwardTaylor2} we get
$$ f(x+h)-f(x-h) = 2h \frac{\partial f}{\partial x}(x) + O(h^2).$$

Therefore we obtain the \textit{central difference} approximation of the first derivative
\begin{equation}\label{cf1}
\frac{\partial f}{\partial x}(x) = \frac{f(x+h)-f(x-h)}{2h} + O(h^2).
\end{equation}

\subsection{Central Difference Approximation of Second Derivative}
%Central Difference Approximation of Second Derivative
Finally summing equations \eqref{forwardTaylor2} and \eqref{backwardTaylor2} we get
$$2f(x) = f(x+h)+f(x-h)+ h^2 \frac{\partial^2 f}{\partial x^2}(x) + O(h^2).$$ 

Then the \textit{central difference} approximation of the second derivative is given by
\begin{equation}\label{cf2}
\frac{\partial^2 f}{\partial x^2}(x) = \frac{f(x+h)-2f(x)+f(x-h)}{h^2}+O(h^2).
\end{equation}

\section{Option Pricing under the Generalized Black-Scholes model}
\label{sec:FD:GBS_PDE}
%Option Pricing under the Generalized Black-Scholes model
Consider the Generalized Black-Scholes model, which includes the \textit{local volatility} $\sigma(S,t)$ and term structures of \textit{interest rate} $r(t)$ and \textit{dividend rate} $q(t)$. The price of an asset $S$ under such model follows the \textit{stochastic differential equation} (SDE):
$$dS_t = (r(t)-q(t))S_t dt + \sigma(S_t,t)S_tdW_t.$$

Then we know that the value of an option $v(S,t)$ on that asset $S$ satisfies the following \textit{partial differential equation} (PDE):
$$\begin{cases}
\frac{\partial v}{\partial t} + \frac{\sigma(S,t)S^2}{2}\frac{\partial^2 v}{\partial S^2}+(r(t)-q(t))S\frac{\partial v}{\partial S}= r(t)v(S,t)\\
v(S,T) = \Psi(S) & \text{Terminal Condition (Payoff function)}\\
\frac{\partial^2 v}{\partial S^2}(S_{\max},t)=\frac{\partial^2 v}{\partial S^2}(S_{\min},t)=0 &\text{Neumann Boundary Conditions}
\end{cases}$$

Now if we use the change of variable $\tau = (T-t)$ to express \textit{time to maturity}, we obtain the following PDE:
$$\begin{cases}
-\frac{\partial v}{\partial \tau} + \frac{\sigma(S,\tau)S^2}{2}\frac{\partial^2 v}{\partial S^2}+(r(\tau)-q(\tau))S\frac{\partial v}{\partial S}= r(\tau)v(S,\tau)\\
v(S,0) = \Psi(S) & \text{Initial Condition (Payoff function)}\\
\frac{\partial^2 v}{\partial S^2}(S_{\max},\tau)=\frac{\partial^2 v}{\partial S^2}(S_{\min},\tau)=0 &\text{Neumann Boundary Conditions}
\end{cases}$$

To begin, we have to define the domain of the problem $$D = \{S_{\min}\leq S\leq S_{\max} ; 0 \leq \tau \leq T\}$$
and set it to a discrete grid
$$\bar{D} = \begin{Bmatrix}
S_j = S_{\min}+(j-1)h; & h=\frac{S_{\max}-S_{\min}}{N}; & j = 1,\ldots,N+1 \\ t_k = 0 + (k-1)\Delta t; & \Delta t = \frac{T}{M}; &k = 1,\ldots,M+1 &
\end{Bmatrix},$$
where $N$ is the number of subintervals in the $S$-direction and $M$ is the number of subintervals in the $\tau$-direction.

\subsection{Forward Euler Approximation}
%Forward Euler Approximation
The Forward Euler approximation constructs the \textit{explicit} discretization of the Generalized Black-Scholes PDE. In other words, we approximate the theta term $\frac{\partial v}{\partial t}(S,t)$ using a \textit{forward difference} approximation \eqref{fwd}:
$$\frac{\partial v}{\partial t}(S,t) \approx \frac{v(S,t+\Delta t)-v(S,t)}{\Delta t}.$$
The \textit{central difference} approximation of the first derivative \eqref{cf1} for the delta term $\frac{\partial v}{\partial S}(S,t)$ gives us
$$\frac{\partial v}{\partial S}(S,t) \approx \frac{v(S+h,t)-v(S-h,t)}{2h}$$
and the \textit{central difference} approximation of the second derivative \eqref{cf2} for the gamma term $\frac{\partial^2 v}{\partial S^2}(S,t)$ gives
$$\frac{\partial^2 v}{\partial S^2}(S,t) \approx \frac{v(S+h,t)-2 v(S,t)+v(S-h,t)}{h^2}.$$

\subsection{Backward Euler Approximation}
%Backward Euler Approximation

\subsection{$\theta$-Method and Crank-Nicolson Approximation}
%Theta-Method and Crank-Nicolson Approximation
